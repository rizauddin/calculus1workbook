% !TEX root = ../main.tex
\section{The Second Fundamental Theorem of Calculus}
 
\begin{myframe}[arc=10pt,auto outer arc]
If $f$ is continuous on interval $I$, $A$ is any number in $I$, $F$ is an antiderivative of $F$ on $I$.
	\begin{enumerate}
\item $\displaystyle F(x) = \int_a^x f(t) \,dt \implies F'(x) = \frac{d}{dx} \int_a^x f(t) \,dt = f(x) $
\item $\displaystyle F(x) = \int_a^{g(x)} f(t) \,dt \implies F'(x) = \frac{d}{dx} \int_a^{g(x)} f(t) \,dt = f(g(x)) g'(x) $
\end{enumerate}
\end{myframe}

\pairofprobsans%
{$\displaystyle \frac{d}{dx} \int_1^x t^3 \,dt $}{$\displaystyle x^3$}%
{$\displaystyle \frac{d}{dx} \int_1^x \frac{\sin{(t)}}{t} \,dt $}{$\displaystyle \frac{\sin{(x)}}{x}$}%

\problemans%
{$\displaystyle \frac{d}{dx} \int_2^{3x^2} 4u \,du $}{$\displaystyle 72x^3$}%

%%%% GUIDES
\qrfigure{chapter4/qr/The-Second-Fundamental-Theorem-of-Calculus}{Scan for guides}

