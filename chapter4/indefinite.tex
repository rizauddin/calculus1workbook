% !TEX root = ../main.tex

\section{The Indefinite Integral}

\begin{myframe}[arc=10pt,auto outer arc]
\[ \int b \,dx = bx + C
\]
\end{myframe}

\pairofprobsans%
{$\displaystyle \int 4 \,dx$}{$4x + C$}%
{$\displaystyle \int 0.4 \,dy$}{$0.4y + C$}%

\pairofprobsans%
{$\displaystyle \int \frac{1}{3} \,dx$}{$\displaystyle \frac{x}{3} + C$}%
{$\displaystyle \int e \,d\theta$}{$\displaystyle e \theta +C$}%

\pairofprobsans%
{$\displaystyle \int \sqrt{2} \,dx$}{$\displaystyle \sqrt{2}x + C$}%
{$\displaystyle \int (3 + \sqrt{2}) \,dz$}{$\displaystyle (3 + \sqrt{2})z + C$}%

\problemans%
{$\displaystyle \int \pi \,dx$}{$\pi x +C$}%

%-----------------------------
\newpage
\begin{myframe}[arc=10pt,auto outer arc]
	\[ \int x^r \,dx = \frac{x^{r+1}}{r + 1} + c; r\ne -1
	\]
\end{myframe}

\pairofprobsans%
{$\displaystyle \int x \,dx$}{$\displaystyle \frac{x^2}{2} + C$}%
{$\displaystyle \int 4x \,dx$}{$2x^2 + c$}%

\pairofprobsans%
{$\displaystyle \int x^2 \,dx$}{$\displaystyle \frac{x^3}{3} + C$}%
{$\displaystyle \int 3x^2 \,dx$}{$x^3 + c$}%

\pairofprobsans%
{$\displaystyle \int x^3 \,dx$}{$\displaystyle \frac{x^4}{4} + C$}%
{$\displaystyle \int 0.5x^3 \,dx$}{$\displaystyle \frac{x^4}{8} + C$}%

\problemans%
{$\displaystyle \int x^{0.5} \,dx$}{$\displaystyle \frac{x^{1.5}}{1.5} + C$}%

%-----------------------------

\newpage
\begin{myframe}[arc=10pt,auto outer arc]
	\[ \int x^r \,dx = \frac{x^{r+1}}{r + 1} + c; r\ne -1
	\]
\end{myframe}

\problemans%
{$\displaystyle \int (x + x^2) \,dx$}{$\displaystyle \frac{x^2}{2} + \frac{x^3}{3} + C$}%

\problemans%
{$\displaystyle \int (3x^6 - 2x^2 + 7x + 1) \,dx$}{$\displaystyle 3\frac{x^7}{7} - 2\frac{x^3}{3} + 7\frac{x^2}{2} + x + C$}%

%-----------------------------

\newpage
\begin{myframe}[arc=10pt,auto outer arc]
	\[ \int x^r \,dx = \frac{x^{r+1}}{r + 1} + c; r\ne -1
	\]
\end{myframe}

\pairofprobsans
{$\displaystyle \int \sqrt{x} \,dx$}{$\displaystyle \frac{2}{3}x^{\frac{3}{2}}+ C$}%
{$\displaystyle \int \sqrt[3]{x} \,dx$}{$\displaystyle \frac{3}{4} x^{\frac{4}{3}} + C$}%

\problemans%
{$\displaystyle \int \sqrt{x^3} \,dx$}{$\displaystyle \frac{2}{5} x^{\frac{5}{2}} + C$}%

%-----------------------------


\newpage
\begin{myframe}[arc=10pt,auto outer arc]
	\[ \int x^r \,dx = \frac{x^{r+1}}{r + 1} + c; r\ne -1
	\]
\end{myframe}

\pairofprobsans%
{$\displaystyle \int (x+2)(x-3) \,dx$}{$\displaystyle \frac{x^3}{3} - \frac{x^2}{2} -6x + C$}%
{$\displaystyle \int \frac{x^3 + 2x^2}{x} \,dx$}{$\displaystyle \frac{x^3}{3} + x^2 + C$}%

\pairofprobsans%
{$\displaystyle \int \frac{1}{x^3}  \,dx$}{$\displaystyle -\frac{1}{2x^2} + C$}%
{$\displaystyle \int \frac{3}{x^2} \,dx$}{$\displaystyle -\frac{3}{x} + C$}%

%-----------------------------


\newpage
\begin{myframe}[arc=10pt,auto outer arc]
	\begin{enumerate}
	\item $\displaystyle \int x^r \,dx = \frac{x^{r+1}}{r + 1} + c; r\ne -1$
	\item $\displaystyle \int \frac{1}{u} \, du = \ln{u} + C$
	\item $\displaystyle \int e^u \, du = e^u + C$
	\item $\displaystyle \int b^u \, du = \frac{b^u}{\ln{b}} + C$
	\end{enumerate}
\end{myframe}

\pairofprobsans%
{$\displaystyle \int \frac{2}{x} \,dx$}{$\displaystyle 2\ln{x} + C$}%
{$\displaystyle \int 5e^x \,dx$}{$\displaystyle 5e^x + C$}%

\pairofprobsans%
{$\displaystyle \int 2^x \,dx$}{$\displaystyle \frac{2^x}{\ln{2}} + C$}%
{$\displaystyle \int \pi^x \,dx$}{$\displaystyle \frac{\pi^x}{\ln{\pi}} + C$}%

%-----------------------------


\newpage
\begin{myframe}[arc=10pt,auto outer arc]
	\begin{enumerate}
		\item $\displaystyle \int \sin{(x)} \,dx = -\cos{(x)} + C$
		\item $\displaystyle \int \cos{(x)} \,dx = \sin{(x)} + C$
		\item $\displaystyle \int \sec^2{(x)} \,dx = \tan{(x)} + C$
		\item $\displaystyle \int \csc^2{(x)} \,dx = -\cot{(x)} + C$
		\item $\displaystyle \int \sec{(x)} \tan{(x)} \,dx = \sec{(x)} + C$
		\item $\displaystyle \int \csc{(x)} \cot{(x)} \,dx = -\csc{(x)} + C$
	\end{enumerate}
\end{myframe}


\pairofprobsans%
{$\displaystyle \int 2\sin{(x)} \,dx$}{$\displaystyle -2\cos{(x)} + C$}%
{$\displaystyle \int 10\cos{(x)} \,dx$}{$\displaystyle 10\sin{(x)} + C$}%

\pairofprobsans%
{$\displaystyle \int 5\sec^2{(x)} \,dx$}{$\displaystyle 5\tan{(x)} + C$}%
{$\displaystyle \int 2\csc^2{(x)} \,dx$}{$\displaystyle -2\cot{(x)} + C$}%


\pairofprobsans%
{$\displaystyle \int 10 \sec{(x)} \tan{(x)} \,dx$}{$\displaystyle 10\sec{(x)} + C$}%
{$\displaystyle \int 2\csc{(x)} \cot{(x)} \,dx$}{$\displaystyle -2\csc{(x)} + C$}%
%-----------------------------

%%%% GUIDES
\qrfigure{chapter4/qr/The-Indefinite-Integrals}{Scan for guides}
